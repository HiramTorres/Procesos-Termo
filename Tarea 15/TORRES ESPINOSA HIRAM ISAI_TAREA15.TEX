\documentclass{article}
\usepackage[utf8]{inputenc}
\usepackage[spanish]{babel}
\usepackage{graphicx}

\title{TAREA 15}
\author{Hiram Torres }
\date{31 de Marzo 2022}

\begin{document}

\maketitle

\begin{enumerate}
    \item ¿Por qué elementos está constituido el sistema biela – manivela deslizante en un motor diésel?
        \begin{itemize}
            \item Es muy similar al de gasolina, por el sistema biela-manivela, el émbolo, el cilindro 
        \end{itemize}
    \item ¿Qué función tiene la culata del cilindro en un motor diésel?
        \begin{itemize}
            \item  Es el encargado de evitar pérdidas de compresión en el interior de los cilindros, cerrando la parte superior de estos.
        \end{itemize}
    \item Mecánicamente en un motor diésel ¿qué ocurre en el Tiempo de Succión y Retroceso? Explica brevemente
        \begin{itemize}
            \item En el tiempo de succión la válvula de entreda se abre permitiendo pasar el combustible en mezcla con oxigeno, para asi dar al paso del resoceso que es cuando las válvulas se cierran y el pistón sube para llegar al tiempo de compresión.  
        \end{itemize}
    \item Mecánicamente en un motor diésel ¿qué ocurre en el Tiempo de Compresión? Explica brevemente 
        \begin{itemize}
            \item En el tiempo de compresión el pistón sube y las válvulas se cierran haciendo que se comprima el combustible, haciendo que se eleve la temperatura y la presión para que se produzca una explosión. Haciendo asi que el pistón baje y que transforme la energía calorifica en energía mecánica 
        \end{itemize}
    \item Mecánicamente en un motor diésel ¿qué ocurre en el Tiempo de Potencia? Explica brevemente
        \begin{itemize}
            \item Es cuando después de la compresión, el pistón baja a causa de la energía liberada por la explosión y así convirtiendolo en energía mecánica.
        \end{itemize}
    \item Mecánicamente en un motor diésel ¿qué ocurre en el Tiempo de Escape? Explica brevemente
        \begin{itemize}
            \item En este tiempo, el pistón vuelve a subir y la válvula de escape abre, haciendo así que el pistón expulse el vapor ocasionado por la explosión y se complete el tiempo de escape. 
        \end{itemize}
        
    \item ¿Cuál es la utilidad del “bowl” que se encuentra en la parte superior del cilindro del motor diésel?
        \begin{itemize}
            \item El bowl sirve para que cuando el pistón suba, el aire se arremoline junto con el combustible y se produzca una mezcla más efectiva. 
        \end{itemize}
    \item ¿Por qué en los motores diésel hay mayor vibración y ruido en comparación con los motores de gasolina?
    ¿Cómo se elimina esa vibración?
        \begin{itemize}
            \item Por que el proceso de combustión no es uniforme siempre y tampoco es suave, debido a esto el motor es más ruidoso y también con más vibraciones. Y esto se soluciona teniendo más cilindros, debido a que genera una uniformidad aunque no todos los cilindros estén en el mismo tiempo . 
        \end{itemize}
    \item ¿Cuál es el orden de explosión en un motor diésel de cuatro cilindros?
        \begin{itemize}
            \item El orden es 1-3-4-2 
        \end{itemize}
    \item Explica la funcionalidad de los árboles de levas en un motor diésel
        \begin{itemize}
            \item Sirven para cerrar y abrir las valvulas de entrada y escape de una forma precisa. Siendo asi que los arboles de levas se muevan a la mitad de tiempo que el cigüeñal. 
        \end{itemize}
    \item ¿A qué rendimiento se acerca el motor Stirling?
        \begin{itemize}
            \item Se acerca mucho al rendimiento teórico de carnot 
        \end{itemize}
    \item ¿Cuáles son las características generales y funcionamiento del motor Stirling? Explica brevemente
        \begin{itemize}
            \item El funcionamiento del motor stirling se basa en la expansión y la contracción cíclica de un gas sellado en su interior. Dicha expansión y contracción se consigue exponiendo el gas una y otra vez a una fuente de calor y a otra de frío, respectivamente.
        \end{itemize}
    \item ¿Cuáles son las razones por las cuáles la utilidad del motor Stirling es muy reducida en aplicabilidad?
        \begin{itemize}
            \item Debido a que su rendimiento optimo solo se alcanza a velocidades bajas y aunque es muy amigable con el medio ambiemte, no es util para la vida laboral. 
        \end{itemize}
    \item Explica brevemente los procesos $1 \rightarrow 2, 2  \rightarrow 3,3 \rightarrow 4$ y $4  \rightarrow 1,$ en un motor Stirling
        \begin{itemize}
            \item Proceso $1 \rightarrow 2$: El cilindro se calienta a través de una fuente de calor, y el gas que está dentro del cilindro tiende a calentarse y por consecuente a expandirse
            \item Proceso $2 \rightarrow 3$: El gas se expande haciendo que el cilindro se mueva hacia la parte inferior y pasando también por la holgura ingresando asi al otro piston y generando asi el trabajo del motor
            \item Proceso $3 \rightarrow 4$: Se disipa el calor del gas al exterior para bajar  la temperatura con la ayuda de las paletas. 
            \item Proceso $4 \rightarrow 1$: Cuando baja la temperatura del gas este se contrae y el volante de inercia gana fuerza haciendo que se retorne el gas sin mas esfuerzo  
            
        \end{itemize}
        
    
    \item ¿Cuáles son las similitudes y las diferencias entre un motor Stirling y un motor Ericsson?
        \begin{itemize}
            \item 
        \end{itemize}
    \item Explica brevemente los procesos $1 \rightarrow 2,2 \rightarrow 3,3 \rightarrow 4 $y$ 4 \rightarrow 1$ en un motor Ericsson
        \begin{itemize}
            \item Proceso $1 \rightarrow 2$: 
            \item Proceso $2 \rightarrow 3$
            \item Proceso $3 \rightarrow 4$
            \item Proceso $4 \rightarrow 1$
        \end{itemize}
    
    \item Define el concepto de regeneración en los ciclos Stirling y Ericsson
        \begin{itemize}
            \item  Actúa como un sistema que almacena energía en cada ciclo. El calor se deposita en el regenerador cuando el gas se desplaza desde el foco caliente hacia el foco frío disminuyendo su temperatura.
        \end{itemize}

\end{enumerate}




\end{document}
