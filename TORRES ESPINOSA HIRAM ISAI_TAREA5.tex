\documentclass[12pt, letterpaper]{article}
\usepackage[spanish]{babel}
\usepackage[utf8]{inputenc}
\usepackage{blindtext}
\usepackage{multicol}
\usepackage{soul}
\setlength{\columnsep}{2cm}

\title{Tarea 5}
\author{Hiram I. Torres Espinosa}
\date{02 Enero 2022}

\begin{document}
    \begin{titlepage}
        \maketitle    
    \end{titlepage}
    

   

\section{Problema 1 Ley de Boyle}
Un globo inflado que tiene un volumen de 0.55 L a nivel del mar (1 atm) se eleva a una altura de
6.5 km, donde la presión es de cerca de 0.4 atm. Suponiendo que la temperatura permanece
constante, ¿cuál es el volumen final del globo?\\ 
\begin{multicols}{2}
\textbf{Datos:}

\begin{itemize}
    \item $V_1 = 0.55 L$
    \item $P_1 = 1 atm (101.3 Kpa)$ 
    \item $V_2 = ?$
    \item $P_2 = 0.4atm (40.53 Kpa)$
\end{itemize}


\textbf{Procedimiento:}\\

$P_1 V_1 = P_2 V_2$ \\ 

{\large$V_2 = \frac{P_1 V_1}{P_2}$}\\ 


{\large $V_2 = \frac{(101.3 Kpa) (0.55 L)}{40.53 Kpa}$
}\\

\ul{$V_2 = 1.375 L$}
\end{multicols}

\section{Problema 2 Ley de Charles}
Una muestra de 452 $mL$ gas flúor se calienta desde 22°$C$ hasta 187°$C$ a presión constante ¿cuál es
su volumen final? 

\begin{multicols}{2}

{\large \textbf{Datos:}}
\begin{itemize}
    \item $V_1 = 0.452 L$
    \item $T_1 =$ 22°$C$ $(295.15 K)$
    \item $V_2 = ?$
    \item $T_2 =$ 187°$C$ $(460.15 K)$
\end{itemize}
\columnbreak
{\large \textbf{Procedimiento:}}\\ \\

{\large
$\frac{V_1}{T_1} = \frac{V_2}{T_2}$}\\ \\

{\large $V_2 = \frac{(V_1)(T_2)}{T_1}$}\\ \\

{\large $V_2 = \frac{(0.452L)(460.15K)}{295.15K}$} \\ \\

\ul{\large $V_2 = 0. 705 L$}

\end{multicols}

\section{Problema 3 Ley de los gases ideales}
El hexafluoruro de azufre (SF6) es un gas incoloro, inodoro y muy reactivo. Calcule la presión (en
atm) ejercida por 1.82 moles del gas en un recipiente de acero de 5.43 L de volumen a 69.5°C\\

\begin{multicols}{2}
{\large \textbf{Datos:}}
\begin{itemize}
    \item $P = ?$
    \item $n = 1.82 mol$
    \item $V = 5.43 L$
    \item $T = 69.5°C (342.65K)$
    \item $R = 0.082 \frac{(L)(atm)}{(K)(mol)}$
\end{itemize}
\columnbreak

{\large \textbf{Procedimiento:}}\\

\large{$PV = n R T$}\\

\large{$P = \frac{nRT}{V}$}\\

\large{$P = \frac{(1.82mol)(0.082\frac{Latm}{Kmol})(342.65K)}{(5.43L)}$}\\

\ul{\large$P = 9.418 atm$}

\end{multicols}

\section{Problema 4 Ley de los gases}
Una pequeña burbuja se eleva desde el fondo de un lago, donde la temperatura y presión son 8°C
y 6.4 atm, hasta la superficie del agua, donde la temperatura es 25°C y la presión es de 1 atm.
Calcule el volumen final (en mL) de la burbuja si su volumen inicial fue de 2.1 mL
\begin{multicols}{2}
\large \textbf{Datos:}
\begin{itemize}
    \item $T_1 = 8°C(281.15K)$
    \item $P_1 = 6.4 atm (648.48 Kpa)$
    \item $T_2 = 25°C (298.15K)$
    \item $P_2 = 1atm (101.325Kpa)$
    \item $V_2 = ?$
    \item $V_1 = 2.1 mL$ 
\end{itemize}
\columnbreak

\large \textbf{Procedimiento:}\\

\large {$\frac{P_1V_1}{T_1}=\frac{P_2V_2}{T_2}$}\\

\large {$V_2 =\frac{P_1V_1T_2}{T_1P_2}$}\\

\large {$V_2 =\frac{(648.48Kpa)(0.0021L)(298.15K)}{(281.15K)(101.325Kpa)}$}\\

\ul {\large $V_2 = 14.25 mL$}
\end{multicols}


\section{Problema 5 Ley de Gay-Lussac}

El argón es un gas inerte que se utiliza en los focos para retrasar la vaporización del filamento. Un
cierto foco tiene argón a 1.2 atm y 18°C se calienta a 85°C a volumen constante. Calcule su presión
final (en atm).

\begin{multicols}{2}
\large \textbf{Datos:}
\begin{itemize}
    \item $P_1 = 1.2 atm$
    \item $T_1 = 18°C (291.15K)$
    \item $P_2 = ?$
    \item $T_2 = 85°C (358.15 K)$
\end{itemize}
\columnbreak


\large \textbf{Procedimiento:}\\

\large $\frac{P_1}{T_1} = \frac{P_2}{T_2}$\\

\large $P_2 = \frac{P_1T_2}{T_1}$ \\

\large $P_2 = \frac{(121.59Kpa)(359.15K)}{(291.15K)}$ \\

\ul {\large $P_2 = 1.476 $}






\end{multicols}




\end{document}